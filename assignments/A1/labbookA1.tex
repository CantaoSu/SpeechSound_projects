%%%%%%%%%%%%%%%%%%%%%%%%%%%%%%%%%%%%%%%%%%%%%%%%%%%%%%%%%%%%%%%%%%%%%%%%%%%%%%%%%%%%
% Do not alter this block (unless you're familiar with LaTeX)
\documentclass{../labbook}
\usepackage{tipa}
%%%%%%%%%%%%%%%%%%%%%%%%%%%%%%%%%%%%%%%%%%%%%
%Fill in the appropriate information below
\lhead{Cantao Su}
\rhead{Speech Sounds} 
\chead{\textbf{Lab Book A1. Due: 13.09.2023 23:59 CEST}}
%%%%%%%%%%%%%%%%%%%%%%%%%%%%%%%%%%%%%%%%%%%%%

\begin{document}

\begin{mdframed}[backgroundcolor=blue!20]
\LaTeX ~submissions are mandatory. Submitting your assignment in another format will be graded no higher than R. Convenient table for the TIPA package used to processing IPA symbols can be found \href{https://jon.dehdari.org/tutorials/tipachart_mod.pdf}{here}. Detailed TIPA manual can be found \href{http://www.l.u-tokyo.ac.jp/~fkr/tipa/tipaman.pdf}{here}.
\end{mdframed}

\tableofcontents %Automatic table of contents

\section{Lab Book A1.}
In this focus on differences between phonemes and orthographic letters. 

\begin{problem}{1}
\textbf{[2]} \textit{Same sound}: Come up with a set of 4 English CVC words with the same vowel, where C = consonant, V = vowel. CVCC, CCVC and CCVCC words are also allowed. For example, “hoot”, “grew”, “moon“, “suit” all have the same vowel \textipa{/u/}. Name the vowel in IPA.
\end{problem}
%%%%%%%%%%%%%%%%%%%%%%%%%%%%%%%%%%%%%%%%%%%%% 

\begin{solution}
Four words with the same vowel: \textbf{near open/low front unrounded vowel /\ae/}
\begin{enumerate}
    \item \underline{cat}
    \item \underline{pad}
    \item \underline{sad}
    \item \underline{mat}
\end{enumerate}


\end{solution}
%%%%%%%%%%%%%%%%%%%%%%%%%%%%%%%%%%%%%%%%%%%%%
\begin{problem}{2}
\textbf{[4]} \textit{Counting sounds}: Give three examples of the English words where there are more sounds than letters and three examples of the English words where there are less sounds than letters.
E.g., in word ``axis'' there are 5 sounds, and in the word ``knock'' there 3 sounds.  Do not use the same examples of the letter-sound combinations, think of different ones (e.g., do not provide three words with letter 'x' that makes the two sounds /ks/). 
\end{problem}
%%%%%%%%%%%%%%%%%%%%%%%%%%%%%%%%%%%%%%%%%%%%%

\begin{solution}
Three words where there are more sounds than letters:
\begin{enumerate}
    \item \underline{xylem}
    \item \underline{languages}
    \item \underline{WC}
    \item \underline{FBI}
\end{enumerate}
\bigskip

Three words where there are less sounds than letters:
\begin{enumerate}
    \item \underline{honour}
    \item \underline{sigh}
    \item \underline{asthma}
\end{enumerate}

\end{solution}
%%%%%%%%%%%%%%%%%%%%%%%%%%%%%%%%%%%%%%%%%%%%%
\begin{problem}{3}
\textbf{[4]} \textit{Be creative}: A famous example of irregularities in English spelling and pronunciation is word “ghoti”, first mention of which dates back to 1855 (in the correspondence between Charles Ollier and Leigh Hunt).

The word is intended to be pronounced in the same way as ``fish'' (\textipa{/fIS/}) using these sounds:
\begin{itemize}
    \item gh, pronounced as \textipa{/f/} as in enough \textipa{/I"n2f/}
    \item o, pronounced as \textipa{/I/} as in women \textipa{/"wImIn/}
    \item ti, pronounced as \textipa{/S/} as in nation \textipa{/"neIS9n/} or motion \textipa{/"moUS9n/}
\end{itemize}


One would expect to pronounce the word in English as "goatee", \textipa{/"goUti/}, not "fish".

Come up with a creative or absurd spelling of any existing word that would demonstrate English or other language idiosyncrasies. Provide \underline{transcriptions for the words} from which you have taken specific sounds \textbf{and} \underline{the expected pronunciation} of your creative spelling. You can follow the “ghoti” example above.

\end{problem}

\begin{solution}

\underline{Your word(s)}: \textbf{quaff}
\bigskip

\underline{Creative or absurd spelling of your word(s)}: \textbf{chauph}
\bigskip

\underline{Transcriptions for the words from which you have taken specific sounds}:
\begin{itemize}
    \item ch, pronounced as /kw/ as in choir /\textprimstress kwa\textsci \textschwa (r)/
    \item au, pronounced as /\textturnscripta/ as in because /b\textsci \textprimstress k\textturnscripta z/
    \item ph, pronounced as /f/ as in phone /f\textschwa \textupsilon n/
\end{itemize}

\underline{The pronunciation of your creative spelling}: what you would expect people to pronounce your made up spelling? \textbf{/kw\textturnscripta f/} 
\bigskip

Cantao Su


\end{solution}
%%%%%%%%%%%%%%%%%%%%%%%%%%%%%%%%%%%%%%%%%%%%%

\end{document}
