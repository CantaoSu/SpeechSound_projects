    %%%%%%%%%%%%%%%%%%%%%%%%%%%%%%%%%%%%%%%%%%%%%%%%%%%%%%%%%%%%%%%%%%%%%%%%%%%%%%%%%%%%
% Do not alter this block (unless you're familiar with LaTeX)
\documentclass{../labbook}
%%%%%%%%%%%%%%%%%%%%%%%%%%%%%%%%%%%%%%%%%%%%%
%Fill in the appropriate information below
\lhead{YOUR NAME HERE}
\rhead{Speech Sounds} 
\chead{\textbf{Lab Book A5. Due: 20.10.2023 23:59 CEST}}
%%%%%%%%%%%%%%%%%%%%%%%%%%%%%%%%%%%%%%%%%%%%%

\begin{document}

\begin{mdframed}[backgroundcolor=blue!20]
\LaTeX ~submissions are mandatory. Submitting your assignment in another format will be graded no higher than R.
\end{mdframed}

%\tableofcontents %Automatic table of contents

\section{Lab Book A5}
In this Lab Book assignment you will investigate the interaction of fundamental frequency and
lexical tone in Mandarin Chinese \footnote{The assignment is inspired by and based on the assignments and materials provided by Dr. Rebecca Scarborough in her Stanford course on phonetics from 2005. 
The data come from the dissertation of Jiahong Yuan, Yuan, J. (2004). Intonation in Mandarin Chinese: Acoustics, perception, and computational modeling. New York: Cornell University, p.416}

In Chinese $f_0$ has at least two linguistic functions: to mark lexical tone, where different pitch patterns distinguish different words, and intonation, when phrases that semantically differ might be distinguished based on pitch. 
But what happens when different pitch requirements of tone and intonation happen to be on a single word?
To investigate this you have 24 sentences in Mandarin Chinese: 4 minimal question / statement sentence pairs produced by 3 speakers. The last word of each sentence has either a high rising tone (tone 2) or a high falling tone (tone 4).


\subsection{Submission}
For this submission you will need to commit the LaTeX file as usual, as well as plots of $f_0$. Therefore, the completed commit for this assignment includes:
\begin{enumerate}
    \item LaTeX file that includes the plots, your hypotheses (e.g., about what you can see on the plots),  statistical results regarding the question, and a very brief discussion of your conclusions.
    \item Your code (you commit your code file alongside your LaTeX file).
    \item Plots.
\end{enumerate}

\begin{problem}{1}{10}{Intonation and tone}

\subsubsection*{Preparation:}

In the dataset (\href{https://drive.google.com/drive/folders/13bNOlbBJdNuvnBxX_OeiHOhqL3jiobXc?usp=sharing}{that can be found here}) the file names are formed as a combination of sentence type (\texttt{st} for statement, \texttt{q} for question), tone, word, speaker. E.g., “\texttt{st\_4\_lu\_S3}” means it is a statement with the fourth-tone word \textit{lu} (``deer'') on the last syllable, produced by Speaker 3. Detailed descriptions are in the tables below.

For the analysis part you should focus your attention on the last word of the sentence. For simplicity you can assume that the last 15\% of each recording correspond to the last word.


\subsubsection*{Task:}
\begin{itemize}
    \item To address the problem you need to track $f_0$ (you can use the Praat script that is committed alongside this Lab Book - I checked it in Windows, works fine with standard Windows paths - or you can try using Robust Algorithm for Pitch Tracking, RAPT, from \href{https://pysptk.readthedocs.io/en/latest/generated/pysptk.sptk.rapt.html}{pysptk}\footnote{If you are going to use RAPT, set hopsize for 80, min $f_0$ for 75, max $f_0$ for 600}). You will have to plot the $f_0$ contours of the sentences and perform simple statistical analysis to answer the following question:
    \textbf{How intonation influences tone realisation in Mandarin Chinese?} 
    
    \item To answer this question, first separately plot $f_0$ for questions and statements with different final tones: tones 2 and 4. \footnote{It is a good idea to normalize $f_0$ by dividing each $f_0$ measurement in each recording by mean $f_0$ of the same recording. This would allow us to ignore the $f_0$ difference between speakers and get nicely aligned pictures. To have the graphs match even better, try dividing the time by the length of the record, so every graph has a ``normalised time'' from 0 to 1.}

    \item  Examine the plots and think of diverse ways of assessing $f_0$ behaviour, such as range, mean, minimum, maximum. Think of hypothesis/hypotheses about $f_0$.
    
    \item  Comparing $f_0$ measurements of questions and statements using a simple ANOVAs or alternatives (e.g., in this task a Kruskal-Wallis test for non-parametric data seems to be a good choice) would be enough for this assignment.\footnote{There are multiple sources you can look into for these tests, a famous and fun book about statistics is by Andy Field, Jeremy Miles and Zoe Field, and you can check it out \href{https://drive.google.com/file/d/1XCNZ9MQ4ZoRNRdMVutQVrtPpWmYfCUJ_/view?usp=sharing}{here}.}
    
    \item Briefly describe the results. Mention any limitations or difficulties you faced during the assignment.

\end{itemize}

You may do this task in whatever programming language you prefer: R, Python, or anything else. If you are struggling with plotting, you may use Praat to draw contours.

If you are using Python, you may consider using \texttt{numpy} or \texttt{pandas} for convenient operations with data sets,  \texttt{scipy.stats} for statistical algorithms and \texttt{matplotlib} for making plots. At the end of the document you can find some Python code snippets to help you if you have chosen Python as a tool for analysis.

\end{problem}

\newcommand{\twolinetext}[2]{
    \vtop{
        \hbox{\strut #1}
        \hbox{\strut (``#2'')}
    }
}

\begin{table}[htpb!]
\caption{\textbf{Statements}}
\begin{tabular}{|l|l|l|}
\hline
File name & Sentence & Literal translation 
\\\hline\hline
    st\_2\_niu    & 
    Luo2yan4 li3bai4wu3 mai4 ye3niu2    &   
    \twolinetext{Luoyan Friday sells wild cows.}{Luoyan sells wild cows on Friday.} \\
\hline
    st\_2\_yang   & 
    Li3bai4wu3 Luo2yan4 yao4 mai3 yang2   & 
    \twolinetext{Friday Luoyan will buy sheep.}{On Friday, Luoyan will buy sheep.} \\
\hline
    st\_4\_lu     & 
    Luo2yan4 li3bai4wu3 mai4 ye3lu4           &   
    \twolinetext{Luoyan Friday sells wild deer.}{Luoyan sells wild deer on Friday.} \\
\hline
    st\_4\_la     & 
    Li3bai4wu3 Luo2yan4 yao4 mai3 la4    & 
    \twolinetext{Friday Luoyan will buy candles.}{On Friday, Luoyan will buy candles.} \\
\hline
\end{tabular}
\end{table}


\begin{table}[htpb!]
\caption{\textbf{Echo questions}}
\begin{tabular}{|l|l|l|}
\hline
File name & Sentence & Literal translation 
\\\hline\hline
    q\_2\_niu &
    Luo2yan4 li3bai4wu3 mai4 ye3niu2 &
    \twolinetext{Luoyan Friday sells wild cows.}{Luoyan sells wild cows on Friday?} \\
\hline
    q\_2\_yang &
    Li3bai4wu3 Luo2yan4 yao4 mai3 yang2 &
    \twolinetext{Friday Luoyan will  buy sheep.}{On Friday, Luoyan will buy sheep?} \\
\hline
    q\_4\_lu &
    Luo2yan4 li3bai4wu3 mai4 ye3lu4 &
    \twolinetext{Luoyan Friday sells wild deer.}{Luoyan sells wild deer on Friday?} \\
\hline
    q\_4\_la &
    Li3bai4wu3 Luo2yan4 yao4 mai3 la4 &
    \twolinetext{Friday Luoyan will buy candles.}{On Friday, Luoyan will buy candles?} \\
\hline
\end{tabular}
\end{table}

    

%%%%%%%%%%%%%%%%%%%%%%%%%%%%%%%%%%%%%%%%%%%%%
\begin{solution}

\subsection{Visualization}

REPLACE THIS TEXT WITH YOUR ANSWER.

\subsection{Hypothesis (e.g., based on the plots)}

REPLACE THIS TEXT WITH YOUR ANSWER.

\subsection{Satistical analysis}

REPLACE THIS TEXT WITH YOUR ANSWER.

\subsection{Brief discussion and conclusion}

REPLACE THIS TEXT WITH YOUR ANSWER.



\end{solution}
%%%%%%%%%%%%%%%%%%%%%%%%%%%%%%%%%%%%%%%%%%%%%



\subsection*{Python tips (for a simplistic solution)}
\begin{enumerate}
    \item Function to read results of pitch tracking from Praat, where ``np'' is imported \texttt{numpy} (\texttt{import numpy as np}): \href{https://gist.github.com/vassverk/b005b5638563b53797624e511a25fb7a}{can be found here}.
    \item Function to measure end tones,  \href{https://gist.github.com/vassverk/b53ae5827b6f4ae9fe41d353a74a9aae}{can be found here}.
    \item Plotting tones together and measuring end tones \href{https://gist.github.com/vassverk/68afaa710d6233695feda2a67612416a}{can be found here}, where:
\begin{itemize}
    \item ``pd'' is imported \texttt{pandas} (\texttt{import pandas as pd}),
    \item ``plt'' (\texttt{from matplotlib import pyplot as plt}),
    \item \texttt{data\_dir = Path('data')},
    \item ``end\_of\_sentence\_fraction'' is a parameter defining which ``end portion'' of the sentence you analyse (it is for you to define).
\end{itemize}
\end{enumerate}

 
\end{document}